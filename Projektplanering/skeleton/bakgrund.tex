\section{Background}

Everyone has probably heard of the term logic in their everyday life and it can be explained shortly as a study of reasoning, inferring conclusions from known or assumed truths, also known as premises. Computer scientists, mathemathicians and psychologists all use different types of logic. Two common types of logic are \textit{propositional logic} and \textit{predicate logic}.

Propositional logic deals with simple statements, propositions. Consider the following sentences, written in natural language:

\begin{enumerate}
      \item ``If Adam is eating an apple then Peter is not happy.''
      \item ``Peter is happy.''
\end{enumerate}

From these sentences, it is possible to conclude that

\begin{enumerate}[resume]
      \item ``Adam is not eating an apple''.
\end{enumerate}

The conclusion is drawn based on the fact that Peter is never happy when Adam is eating an apple. Since Peter is happy, we know that Adam is not eating an apple.

Predicate logic deals with properties and relationships between objects. Consider the following statements about humans and mortality.

\begin{enumerate}[resume]
      \item ``All humans are mortal.''
      \item ``All adults are human.''
\end{enumerate}

Ignoring the inconsistency with the real world (adult lions are not human), the following conclusion can be inferred from the statements above.

\begin{enumerate}[resume]
      \item ``All adults are mortal.''
\end{enumerate}

The sentence ``All adults are human.'' implies that \textit{adults} is a subset of \textit{humans}. Since the property of mortality applies to all humans, it must apply to all adults, hence the conclusion.

Using natural language quickly becomes inefficient when dealing with larger sets of propositions, and it is therefore customary to use a more compact syntax comprising letters, operators, quantifiers and functions. The examples above can then be rewritten as follows.

\begin{enumerate}
      \item $P \rightarrow \lnot Q$
      \item Q
      \item Therefore $\lnot P$
\end{enumerate}

where $P$ means \textit{Adam is eating an apple} and $Q$ means \textit{Peter is happy}. The example from predicate logic can be rewritten in the following way.

\begin{enumerate}[resume]
      \item $\forall x \, [H(x) \rightarrow M(x)]$
      \item $\forall x \, [A(x) \rightarrow H(x)]$
      \item $\forall x \, [A(x) \rightarrow M(x)]$
\end{enumerate}

where $\forall x$ means \textit{for all} $x$, $H(x)$ means $x$ is human, $M(x)$ means $x$ is mortal and $A(x)$ means $x$ is an adult.

By using this short notation, one can create a well structured, formal language for reasoning about logic. This allows for the use of computer programs to facilitate proof constructions and checking proofs for correctness.

One method used for constructing proofs is \textit{natural deduction} which is the process of applying sound rules to premises which are already proven or part of an assumption. \cite{huth_ryan_2018} This can be done using pen and paper, or with the assistance of a computer program.

There are many proof editors available on the internet, including Easyprove \cite{easyprove}, Yoda \cite{Yoda}, FitchJS \cite{rieppel}, Conan \cite{conan} and fitch-checker \cite{klement}. These projects range from general purpose proof editors to applications targeting students of a specific university course or readers of book.