\section{Bakgrund}

Everyone has probably heard of the term logic in their everyday life and it can be explained shortly as a study of reasoning where you infer conclusions from truths, also known as premises.
Computer Scientists, Mathemathicians and Psychologists are one of many groups people that apply logic in their work, and what type of logic they use depend on what they want to achieve. One of the most
common types of logic is Propositional logic and Predicate logic. Here comes two examples that shows what can appear in these logic types and how logical reasoning can be applied to draw conclusions:


Assume we got the following sentences written informally in natural language:

\begin{itemize}
    \item [1] If Adam eats an apple then Peter will be happy.
    \item [2] Peter is happy.
\end{itemize}


 \noindent From these two sentences we can deduce a third one, which is a conclusion:

 \begin{itemize}
    \item [3] Therefore Adam ate an apple.
\end{itemize}

\noindent This is one example of an inference were we from two arguments could deduce the conclusion(a truth) using logical reasoning in propositional logic. In predicate logic a sentence could be written informally in natural language as:

\begin{itemize}
    \item [4] All humans are mortal. 
\end{itemize}

\noindent How can we conclude that this is true? Well, by informal reasoning all humans are mortal but analyzing the logical structure of this sentence, we can see that we have some properties and human objects involved. By precisely talking about
a specific world where objects live and properties related to this world, we can deduce conclusions. In our world known as Earth, all human beings are mortal and thus we can conclude the sentence is true. 

What we have done currently in both of these examples is that we have logically deduced conclusions informally. This is one way of thinking about logic but usually we work in a more
symbolic nature where these sentences can be translated into so called formulas. Once again, using the examples above, we can write them more formally in a symbolic representation as:

\begin{itemize}
    \item [1] $P => Q$
    \item [2] Q
    \item [3] Therefore P
\end{itemize}

\noindent And in the case of Predicate logic:

\begin{itemize}
    \item [4] $ \forall x \, [H(x) \wedge M(x)]$
\end{itemize}


\noindent A question we should ask ourselves is why we want to use this formal notation when reasoning logically? The answer is simply that we want to formalize these informal languages so we can reason about them more precisely using a computer program. This is done by taking formulas as input into a 
program that performs these kind of deductions from a set of premises to conclusions. This, by applying different predefined rules to eventually deduce a conclusion if it is possible. This is called natural 
deduction which is a calculus of reasoning giving a proof of validity of a potential conclusion.

Our aim of the project is to develop a program which is a natural deduction proof editor for both propositional logic and predicate logic. The editor is going to be using an interactive webb-GUI interface where users can write these kind of formulas to the editor and different predefinied rules in our calculus applied to them to infer new formulas until
a proof is given and a conclusion is found. The program will also handle errors generated by wrong inputs from the user, namely if the formulas are not well definied or if a bad rule is applied during a specific step of the proof. Hints will also be given to the users during a proof deduction if they get stuck or want to see possible deduction steps that can be applied.

Our finished product can be used by all people interested in logic, but the product will be designed with a focus based on assumptions that basic knowledge of propositional logic and predicate logic is known by the user using the editor. 