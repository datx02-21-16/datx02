\section{Problem}

When designing the application, there are several factors to consider. From the user's perspective, easy usage is important, while the developer is more concerned with code quality and maintainability.

\subsection{Usability}
The application is meant to be an educational tool, aimed at students learning natural deduction. Therefore it should not require much introduction as this will draw attention from the subject. To guarantee this, the application will adhere to the following principles and requirements.

\begin{itemize}
    \item The application should be accessible from any platform with minimum requirements on the user's computer.
    \item There must always be a clear focus on the natural deduction process.
    \item Intuitive use and a steep learning curve are more important factors than maximising potential productivity.
\end{itemize}

Furthermore, the application should provide clear advantages compared to using pen and paper. To achieve this, there are some minimum features which must be implemented.

\begin{itemize}
    \item The proof editor should be able to validate syntax.
    \item The proof editor should be able to validate each step in a proof. If a step is invalid, the proof editor should give the user a clear error message, with information about the type of error.
    \item The proof editor should guide students through the construction of proofs, for instance by showing which proof rules can be applied in the next step or partially filling in the next step towards the given goal.
\end{itemize}

Additional possible features are:
\begin{itemize}
    \item The user can save their proof in some convenient format, e.g. to use in a hand-in assignment.
    \item Give the user a choice of exercises to work on. Additionally the exercises could be sorted according to difficulty or filtered to show only those with hints available.
    \item Automate single steps in the proof construction. By choosing an inference rule and existing clauses, the editor can provide the conclusion automatically.
\end{itemize}

\subsection{Code quality}

Intuitive interfaces and good support for the learning process make the application \textit{usable}. Another important part of the development is to make the application \textit{maintainable}, meaning it is possible to handle changing requirements or updates in dependency libraries without rebuilding from the start. An unmaintainable application is inevitably going to break or become obsolete. To ensure as good maintainability as possible, it is important to use good practices for software architecture, documentation and code style.
