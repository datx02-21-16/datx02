\section{Problem}
%
We are creating a proof editor for natural deduction to help students practice constructing proofs.
%
We will make our proof editor in the form of a web application.
%
We believe that this will make it more accessible and easier to try out without commitment.
%
The users will not need to make any changes to their system and can get started fast. \\
%
Web applications that serve as an educational tool for logic has been successfully done before by other projects such as Easyprove \cite{easyprove}, Yoda \cite{Yoda}, FitchJS \cite{rieppel} and fitch-checker \cite{klement}. 
%
These project sometimes had particular target audiences in mind, from students taking a particular course to serving as an accompaniment to a certain book.
%
Our proof editor will first and foremost target students at Chalmers University of Technology and Gothenburg University, and secondly to serve as an accompaniment to the book on logic in computer science by Michael Huth and Mark Ryan \cite{huth_ryan_2018}. \\ \\



Our first task when starting out with our project will be to find out how students in our target group currently go about constructing proofs.
%
We want to find out about their work process and what syntax they use.
%
Getting to know our target group better will help us make a proof editor that most students will find intuitive to use.
%
The result of the study will inform our choice of a graphical user interface, the syntax and how user interaction should work. \\ \\


%
Some functionality we intend to provide with our proof editor are:
%

\begin{itemize}
    \item Validation of steps

%
The proof editor should be able to validate a step, i.e. an application of a proof rule. 
%

%
If a step is invalid, the proof editor should give the user an error message prompting the user to fix the mistake.
%


%
\item Hints

%
The proof editor could help students out with constructing proofs e.g. by showing which proof rules they could apply in the next step or partially filling in the next step towards the given goal. \\


%
Additional possible features are:

%
\item Saving a proof

%
The user can save their proof in some convinient format, e.g. to use in a hand-in assignment.
%

%
\item Providing exercises

%
Give the user a choice of exercises to work on. 

%
Additionally the exercises could be sorted according to difficulty or filtered to show only those with hints available.
%
\end{itemize}